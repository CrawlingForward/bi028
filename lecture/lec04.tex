\documentclass[red]{beamer}

\mode<presentation>{
%\usetheme{Goettingen}
\usetheme{Madrid}
\usecolortheme{default}
}

\usepackage{CJK}
\usepackage{graphicx}
\usepackage{amsmath, amsopn}
\usepackage[english]{babel}
\usepackage[latin1]{inputenc}
\usepackage{url}
\usepackage{color}
\usepackage{times}
\usepackage{xcolor}
\usepackage[formats]{listings}
\lstset{
language=bash,
keywordstyle=\color{blue!70}\bfseries,
basicstyle=\color{black}\ttfamily\scriptsize,
commentstyle=\color{red}\ttfamily,
showspaces=false,
showstringspaces=false,
tabsize=3,
showtabs=true,
columns=fullflexible,
frame=true,
rulesepcolor=\color{red!20!green!20!blue!20},
breaklines=true,
numbersep=5pt}

\lstdefineformat{awk}
{
  \{=\newline\string\newline\indent,%
  \}=\newline\noindent\string\newline,%
  ;=[\ ]\string\space,%
}

\newcommand*{\caret}{%
  \begingroup
    \fontencoding{T1}%
    \fontfamily{qcr}% TeX Gyre Cursor
    %\fontfamily{pcr}% Courier
    \selectfont
    \string^%
  \endgroup
}
\lstdefinestyle{caret}{
  basicstyle=\ttfamily\color{blue},
  literate={^}{\caret}{1},
}
\newcommand*{\lstverb}{\lstinline[style=caret]}

\setbeamertemplate{itemize/enumerate/description body begin}{\footnotesize}
\setbeamertemplate{itemize/enumerate/description subbody begin}{\scriptsize}

\title[BI296-Lec04]{\tiny{BI296: Linux and Shell Programming}\\
\Large{Lecture 04: Bash Scripting}}
\author[Maoying Wu]{Maoying,Wu\\
{\scriptsize ricket.woo@gmail.com}}
\institute[CBB] % (optional, but mostly needed)
{
  \inst{}
  Dept. of Bioinformatics \& Biostatistics\\
  Shanghai Jiao Tong University
}
\date{Spring, 2017}

\AtBeginSubsection[]
{
  \begin{frame}<beamer>{Next we will talk about ...}
    \tableofcontents[currentsection,currentsubsection]
  \end{frame}
}

\newcommand{\tabincell}[2]{\begin{tabular}{@{}#1@{}}#2\end{tabular}}

\begin{document}
\begin{CJK*}{UTF8}{gbsn}
\frame{\titlepage}

\begin{frame}[fragile,containsverbatim]
\frametitle{Lecture Outline}
\begin{itemize}
	\item Bash Programming (BASH脚本编程)
	\begin{itemize}
		\item Bash: Introuction (BASH发展历史与相关概念)
		\item Types of Variables (变量声明与定义)
		\item Conditional Statements and Flow Control(条件结构和流程控制)
		\item Command Line: Arguments (命令行)
		\item BASH Functions (函数的定义)
		\item Built-in Variables and Functions (内置变量与函数)
	\end{itemize}
	\vspace{0.2in}
	\item Applications of BASH(BASH应用案例)
\end{itemize}
\end{frame}

\section{bash: an introduction}

\begin{frame}[fragile,containsverbatim]
\frametitle{The first BASH scripts}
\begin{block}{\centering greeting.sh}
\begin{lstlisting}[language=bash]
#!/bin/bash
# This is the first bash implementation.
# Scriptname: greetings.sh
# Written by: Ricky Woo (ricket.woo@gmail.com)
echo -e "What's your name and your age: "
read name age
echo -e "I'm $name, and I'm $age years old."
read
\end{lstlisting}
\end{block}
\begin{block}{\centering Run the script file}
\begin{lstlisting}[language=bash]
# grant the executable permission
chmod u+x greeting.sh
# run the script
./greeting.sh
bash greeting.sh
\end{lstlisting}
\end{block}
\end{frame}

\begin{frame}[fragile,containsverbatim]
\frametitle{Interpretation of BASH scripts}
\begin{block}{\centering greeting.sh}
\begin{lstlisting}[language=bash]
#!/bin/bash
# This is the first bash implementation.
# Scriptname: greetings.sh
# Written by: Ricky Woo (ricket.woo@gmail.com)
echo -e "What's your name and your age: "
read name age
echo -e "I'm $name, and I'm $age years old."
read
\end{lstlisting}
\end{block}
\begin{block}{\centering Interpretation}
\begin{itemize}
	\item \lstinline{#!/bin/bash}: Shebang to indicate the command 
		to interpret the following scripts.
	\item \lstinline{#}: Comments (注释).
	\item \lstinline{$name}: Variable substitution (变量替换).
	\item \lstinline{read}: Get the variable value interactively 
		(交互式变量赋值). 
\end{itemize}
\end{block}
\end{frame}

\begin{frame}
\frametitle{Supporting Shells: Types and Descriptions}
\begin{itemize}
	\item The default shell for each user is defined in 
		\texttt{/etc/passwd}.
	\vspace{0.2in}
	\item Described in the file \texttt{/etc/shells}.
	\begin{itemize}
		\item \texttt{sh}: Bourne Shell, a light-weighted shell (UNIX)
		\item \texttt{ksh}: Korn shell, superset of \texttt{sh}
		\item \texttt{csh}: Berkeley UNIX C-shell
		\item \texttt{tcch}: An enhanced but completely compatible 
			version of \texttt{csh}.
		\item \texttt{bash}: Bourne Again SHell, superset of \texttt{sh}, 
			\texttt{csh}
	\end{itemize}
\end{itemize}
\end{frame}

\begin{frame}[fragile,containsverbatim]
\frametitle{BASH}
\begin{itemize}
	\item Bourne Again SHell
	\item In memory of Stephen Bourne (Author of Bourne Shell)
	\item \emph{de facto} standard for shell scripting on most 
		flavors of UNIX
	\item \texttt{sh}-compatible
	\item Derive some useful features from \texttt{ksh} as well 
		as \texttt{csh} 
\end{itemize}
\begin{block}{\centering When not to use bash}
\begin{itemize}
	\item Resource-intensive tasks (sorting, hashing, recursion, ...)
	\item Heavy-duty math operations (floating-point arithmetic)
	\item Precision calculations (use c++/fortran instead)
	\item Cross-platform portability required (c/java instead)
\end{itemize}
\end{block}
\end{frame}


\begin{frame}[fragile,containsverbatim]
\frametitle{Variables: Declaration (变量声明)}
\begin{block}{\centering declare: syntax (语法)}
\lstinline{declare [-aAfFgilrtux] [-p] [name[=value] ...]}
\begin{itemize}
\footnotesize
	\item \texttt{-a/-A}: indexed/associative array (索引/关联数组); 
		\texttt{-f/-F}: function body/name (函数实体/函数名); \texttt{i}: integer (整数)
	\item \texttt{-l/-u}: lowercase/uppercase (小写/大写字母)
	\item \texttt{-r}: read-only (只读); \texttt{-g}: global (全局变量)
\end{itemize}
\end{block}
\begin{block}{\centering declare: examples (示例)}
\begin{lstlisting}
# `int` is an integer variable
declare -i int
# `arr` is an indexed array
declare -a arr
# `Arr` is an associative array
declare -A Arr
# `ro` is a read-only variable
declare -r ro
# `func` is a function
declare -f func
\end{lstlisting}
\end{block}
\end{frame}


\begin{frame}[fragile,containsverbatim]
\frametitle{Variables: Definitions (变量定义)}
\centering{\texttt{\underline{name}\textcolor{red}{=}\underline{value}}}
\begin{block}{\centering Variable Names (变量命名)}
\begin{itemize}
	\item \underline{name}=\lstverb{/^[A-Za-z_][A-Za-z0-9_]*$/}
	\item \textcolor{red}{Use interpretable variable names.}
\end{itemize}
\end{block}
\begin{block}{\centering Variable Assignment (变量赋值)}
\begin{itemize}
	\item No space is allowed akin to the \textcolor{red}{=} operator.
	\item \textcolor{red}{Use double quotes to enclose the \underline{value}}.
\end{itemize}
\end{block}
\begin{block}{\centering Examples (变量定义的例子)}
\begin{lstlisting}
# `int` is an integer, 2
declare -i int="8/3"
echo ${int}
\end{lstlisting}
\end{block}
\end{frame}


\begin{frame}[fragile,containsverbatim]
\frametitle{Built-in Variables (内置变量)}
\begin{table}[ht]
\scriptsize
\renewcommand\arraystretch{1.6}
\begin{tabular}{lll}
\hline
\textbf{Variables} & \textbf{Description}\\
\hline
\lstverb|$PATH| & the directories of executable commands.\\
\lstverb|$MANPATH| & the directories of manual pages.\\
\lstverb|$FS| & the field separator, default " ".\\
\lstverb|$PS1, $PS2, $PS3| & the prompt.\\
\lstverb|$PWD, $OLDPWD| & the current/previous working directory.\\
\lstverb|$SHELL| & the default shell for current user.\\
\lstverb|$USER, $USERNAME, $LOGNAME| & the current user name.\\
\lstverb|$HOSTNAME| & the MACHINE name.\\
\lstverb|$HOME| & the home directory for the current user.\\
\lstverb|$?| & the return code of the last command issued.\\
\hline
\end{tabular}
\end{table}
\end{frame}


\begin{frame}[fragile,containsverbatim]
\frametitle{Variable Manipulation}
\begin{block}{}
\begin{lstlisting}
FILEPATH=/path/to/my/output.lis
echo $FILEPATH
echo ${FILEPATH%t*}  # non-greedy tail-truncation
echo ${FILEPATH%%t*} # greedy tail-truncation
echo ${FILEPATH#*/}  # non-greedy head-truncation
echo ${FILEPATH##*/} # greedy head-truncation
echo ${x-5}; echo ${x:-5} # set default value
echo ${x=3}; echo ${x:=3} # set default value
echo ${x+7}; echo ${x:+7} # set default value
echo ${t:?undefined}
dna="ACCTAGGACG"; echo ${dna:3:4} # substring
\end{lstlisting}
\end{block}
\end{frame}


\begin{frame}[fragile,containsverbatim]
\frametitle{Exercise: Variables}
\begin{enumerate}
	\item Which of the following variable names are not legal? If not, tell why.
	\begin{itemize}
		\item \lstinline{3x5, x=1, x+, wt-5, _xyz, #tt, CMT, echo, if, $test}
	\end{itemize}
	\vspace{0.2in}	
	\item Output the results.
	\begin{lstlisting}
#!/bin/bash
num1=3;num2=5
sum=$num1+$num2; echo ${sum} # sum?
declare -i sum=$num1+$num2; echo ${sum} # sum?
diff=$num2-$num1; echo ${diff} # diff?
declare -i diff=$num2-$num1; echo ${diff} # diff?
prod=$num1*$num2; echo ${prod} # product?
declare -i prod=$num1*$num2; echo ${prod} # product?
div=$num2/$num1; echo ${div} # division?
declare -i div=$num2/$num1; echo ${div} # division?
	\end{lstlisting}
\end{enumerate}
\end{frame}

\section{flow control}

\subsection{conditional statements}

\begin{frame}[fragile,containsverbatim]
\frametitle{Testing Files (文件的检验)}
\begin{table}[ht]
\tiny
\renewcommand\arraystretch{1.2}
\begin{tabular}{lll}
\hline
\textbf{Expression} & \textbf{Description}\\
\hline
\lstverb|[ -a FILE ]| & TRUE if FILE exists (存在性).\\
\lstverb|[ -b FILE ]| & TRUE if FILE is a block device file (块设备).\\
\lstverb|[ -c FILE ]| & TRUE if FILE is character device file (字符设备).\\
\lstverb|[ -d FILE ]| & TRUE if FILE is a directory (目录).\\
\lstverb|[ -e FILE ]| & TRUE if FILE exists (存在).\\
\lstverb|[ -f FILE ]| & TRUE if FILE is a regular file (普通文件).\\
\lstverb|[ -g FILE ]| & TRUE if FILE SGID bit is set (SGID).\\
\lstverb|[ -h FILE ]| & TRUE if FILE is a symbolic link (符号链接).\\
\lstverb|[ -k FILE ]| & TRUE if FILE sticky bit is set (粘附位).\\
\lstverb|[ -p FILE ]| & TRUE if FILE is a named pipe (FIFO,管道文件).\\
\lstverb|[ -r FILE ]| & TRUE if FILE is readable (可读).\\
\lstverb|[ -s FILE ]| & TRUE if FILE size is greater than 0 (非空).\\
\lstverb|[ -t FD ]| & TRUE if FILE DESCRIPTOR FD refers to a terminal (终端).\\
\lstverb|[ -u FILE ]| & TRUE if FILE SUID bit is set (SUID).\\
\lstverb|[ -w FILE ]| & TRUE if FILE is writable (可写).\\
\lstverb|[ -x FILE ]| & TRUE if FILE is executable (可执行).\\
\lstverb|[ FILE1 -nt FILE2 ]| & TRUE if FILE1 is newer than FILE2 (新).\\
\lstverb|[ FILE1 -ot FILE2 ]| & TRUE if FILE1 is older than FILE2 (旧).\\
\lstverb|[ FILE1 -ef FILE2 ]| & TRUE if FILE1 and FILE2 refers to the same (同).\\
\hline
\end{tabular}
\end{table}
\end{frame}


\begin{frame}[fragile,containsverbatim]
\frametitle{Test of strings and numbers (字符串和数值的检验)}
\begin{table}[ht]
\tiny
\renewcommand\arraystretch{1.2}
\begin{tabular}{lll}
\hline
\textbf{Expression} & \textbf{Description}\\
\hline
\lstverb|[ -o OPTNAME ]| & TRUE if option OPTNAME is enabled (选项).\\
\lstverb|[ -z STRING ]| & TRUE if STRING is of zero-length (空串).\\
\lstverb|[ -n STRING ]| & TRUE if STRING is non-null (非空).\\
\lstverb|[ STRING ]| & TRUE if STRING is of non-zero length (非空).\\
\lstverb|[ STRING1 == STRING2 ]| & TRUE if two strings are equal (相等).\\
\lstverb|[ STRING1 != STRING2 ]| & TRUE if two strings are not equal (不等).\\
\lstverb|[ STRING1 < STRING2 ]| & TRUE if STRING1 is lexically less than STRING2.\\
\lstverb|[ STRING1 > STRING2 ]| & TRUE if STRING1 is lexically greater than STRING2.\\
\lstverb|[ ARG1 op ARG2 ]| & Arithmetic binary comparison (数值比较).\\
& \texttt{op=(-eq | -ne | -lt | -le | -ge | -gt)}\\
\lstverb|[ !EXPR ]| & TRUE if EXPR is FALSE (逻辑反).\\
\lstverb|[ (EXPR) ]| & Returns the value of EXPR.\\
\lstverb|[ EXPR1 -a EXPR2 ]| & Logical AND (逻辑与).\\
\lstverb|[ EXPR1 -o EXPR2 ]| & Logical OR (逻辑或).\\
\hline
\end{tabular}
\end{table}
\end{frame}

\subsection{if statement}

\begin{frame}[fragile,containsverbatim]
\frametitle{Conditional Statement: \texttt{if}}
\begin{block}{\centering Syntax}
\begin{lstlisting}
if [ cond_statement1 ]; then
  do_something
elif [ cond_statement2 ]; then
  do_other_thing
else
  do_else_thing
fi
\end{lstlisting}
\end{block}
\begin{block}{\centering Example}
\begin{lstlisting}[basicstyle=\tiny\ttfamily,indentation=true]
#!/bin/bash
read -p "Please input a score: " score
if [ $score -gt 90 ]; then
  echo "You got A"
elif [ $score -gt 75 ]; then
  echo "You got B"
elif [ $score -gt 60 ]; then
  echo "You got C"
else
  echo "You failed"
fi
\end{lstlisting}
\end{block}
\end{frame}


\begin{frame}[fragile,containsverbatim]
\frametitle{Three Conditional Expressions}
\begin{table}[ht]
\tiny
\renewcommand\arraystretch{1.6}
\begin{tabular}{lccc}
\hline
 & \lstverb|[ EXPR ]| & \lstverb|[[ EXPR ]]| & \lstverb|test EXPR|\\
\hline
\textbf{Word splitting} & Yes & No & Yes\\
\textbf{Pathname expansion} & Yes & No & Yes\\
\textbf{Pattern globs} & No & Yes & No\\ 
\hline
\end{tabular}
\end{table}
\begin{columns}[t]
\column{0.50\textwidth}
\begin{block}{\centering\scriptsize work splitting: using double quotes}
\begin{lstlisting}[basicstyle=\tiny\ttfamily]
var="split word";
[ $var == "split word" ]; echo $?  # ERROR
[ "$var" == "split word" ]; echo $? # WORK
[[ $var == "split word" ]]; echo $? # WORK
[[ "$var" == "split word" ]]; echo $? # WORK
test $var == "split word"; echo $? # ERROR
test "$var" == "split word"; echo $? # WORK
\end{lstlisting}
\end{block}
\begin{block}{\centering\scriptsize pathname expansion}
\begin{lstlisting}[basicstyle=\tiny\ttfamily]
[ "./lec4.tex" -ef ./lec4.te* ]; echo $?
[[ "./lec4.tex" -ef ./lec4.te* ]]; echo $?
test "./lec4.tex" -ef ./lec4.te*; echo $?
\end{lstlisting}
\end{block}
\column{0.50\textwidth}
\begin{block}{\centering\scriptsize Pattern globbing}
\begin{lstlisting}[basicstyle=\tiny\ttfamily]
[ "expression" == expr* ]; echo $?
[[ "expression" == expr* ]]; echo $?
test "expression" == expr*; echo $?
\end{lstlisting}
\end{block}
\end{columns}
\end{frame}


\begin{frame}[fragile,containsverbatim]
\frametitle{The Other Conditional Statements}
\begin{block}{\centering \texttt{if..then..elif..then..else..fi}}
\begin{lstlisting}
if test "$(whoami)" == "root"; then
  echo "You are using a privileged account"
  exit 1
fi
\end{lstlisting}
\end{block}
\begin{block}{\centering Equivalent \texttt{\&\&} and \texttt{||} }
\begin{lstlisting}
test "$(whoami)" != "root" && (echo "you are using a non-privileged account"; exit 1)
test "$(whoami)" == "root" || echo "Please verify that you have the super-user privilege."
\end{lstlisting}
\end{block}
\end{frame}


\begin{frame}[fragile,containsverbatim]
\frametitle{case...in...esac construct}
\begin{columns}[t]
\column{0.30\textwidth}
\begin{block}{\centering Syntax}
\begin{lstlisting}
case var in
val1)
  statement1
;;;
val2)
  statement2
;;;
*)
  statement3
;;;
esac
\end{lstlisting}
\end{block}
\column{0.70\textwidth}
\begin{block}{\centering Examples}
\begin{lstlisting}
#!/bin/bash
echo "Enter your favorite color: "
read color
case $color in 
r*|R*)
  echo "Your favorite color is red."
;;;
b*|B*)
  echo "Your favorite color is blue."
;;;
g*|G*)
  echo "Your favorite color is green."
;;;
*)
  echo "The color you entered is invalid."
;;;
esac
\end{lstlisting}
\end{block}
\end{columns}
\end{frame}


\begin{frame}[fragile,containsverbatim]
\frametitle{select construct: menu}
\begin{lstlisting}
#!/bin/bash
PS3="Choose your favorite dish (q to exit): "
select dish in "Roast Duck" "Mapo Tofu" "Sweet and Sour Ribs"
do
  case $dish in 
  R*)
    echo "You may be from Beijing."
  ;;
  M*)
    echo "You may be from Sichuan."
  ;;
  S*)
    echo "You may be from Shanghai."
  ;;
  q)
    break
  ;;
  *)
    echo "Let me guess."
  ;;
  esac
  break
done
\end{lstlisting}
\end{frame}

\subsection{loop statement}

\begin{frame}[fragile,containsverbatim]
\frametitle{while loop}
\begin{block}{\centering Syntax}
\begin{lstlisting}
while condition
do
  do_something
done
\end{lstlisting}
\end{block}
\begin{block}{Example}
\begin{lstlisting}
#!/bin/bash
IFS=":"
while read f1 f2 f3 f4 f5 f6 f7
do
  if [ "$f7" == "/sbin/nologin" ]; then
    echo "$f1 cannot login to the system."
  fi
done < /etc/passwd
\end{lstlisting}
\end{block}
\end{frame}


\begin{frame}[fragile,containsverbatim]
\frametitle{General for loop}
\begin{block}{\centering Syntax}
\begin{lstlisting}
for var in val-list
do
  do_something
done
\end{lstlisting}
\end{block}
\begin{block}{\centering Examples}
\begin{lstlisting}
#!/bin/bash
# for1.sh
declare -i j
for i in `seq 20`
do
  j=$i%5
  if [ $j -eq 0 ]; then
    continue
  fi
  echo $i
done
\end{lstlisting}
\end{block}
\end{frame}

\begin{frame}[fragile,containsverbatim]
\frametitle{C-style for loop}
\begin{block}{\centering Syntax}
\begin{lstlisting}
for ((start-condition;end-condition;loop-action))
do
	do_something
done
\end{lstlisting}
\end{block}
\begin{block}{\centering Examples}
\begin{lstlisting}
#!/bin/bash
declare -i j
for ((i=1;i<=20;i++))
do
  j=$i%5
  if [ $j -eq 0 ]; then
    continue
  fi
  echo $i
done
\end{lstlisting}
\end{block}
\end{frame}


\begin{frame}[fragile,containsverbatim]
\frametitle{Exercise}
\begin{enumerate}
	\item Guess what the following scripts do.
	\begin{lstlisting}
#!/bin/bash
# space.sh: A very simple test for checking disk space.
space=`df -h | awk '{print $5}' | grep % | grep -v Use
| sort -n | tail -1 | cut -d "%" -f1`
alertvalue="80"
if [ "$space" -ge "$alertvalue" ]; then
  echo "At least one disk is nearly full!"
else
  echo "Disk space normal"
fi
	\end{lstlisting}
\end{enumerate}
\end{frame}

\section{mathematical computing}

\subsection{integer computation}

\begin{frame}[fragile,containsverbatim]
\frametitle{Four integer computing}
\begin{enumerate}
	\item \lstverb|declare -i|
	\item \lstverb|(())|
	\item \lstverb|expr|
	\item \lstverb|let|
\end{enumerate}
\begin{block}{\centering Example}
\begin{lstlisting}
# declare
declare -i int
int=13%5
echo $int
#  (())
x=3
((x++))
echo $x
# expr
expr 3 * 5
# let
let "x = x * 3"
echo $x
\end{lstlisting}
\end{block}
\end{frame}

\subsection{float computation}

\begin{frame}[fragile,containsverbatim]
\frametitle{floating computing}
\begin{itemize}
	\item The command \texttt{bc} does help.
\end{itemize}
\begin{block}{\centering example}
\begin{lstlisting}
echo "3.3*4" | bc	# 13.2
echo "2.1^4" | bc # 19.4
echo "scale=2; (2.1^2)^2" | bc # 19.44
echo "scale=1; (2.1^2)^2" | bc # 19.3
echo "scale=2; 1.5*100/2" | bc # 75.00
# computing with function
echo "sqrt(2)" | bc
echo "sqrt(2)" | bc -l
echo "a(1)" | bc -l
echo "scale=3; 4*a(1)" | bc
\end{lstlisting}
\end{block}
\end{frame}

\begin{frame}[fragile,containsverbatim]
\frametitle{expr in string operation}
\begin{block}{}
\begin{lstlisting}
# return the string length
expr length "DNA transcription"
# return the substring
expr substr "DNA transcription" 5 13
# return the index
expr index "DNA transcription" 'c'
# regex match
expr account.doc : doc
expr account.doc : acc
# regex match
expr 1234bcdf : ".*"
# capture by regex
expr abcdefgh : '...\(...\)..'
# capture by regex
expr account.doc : '\(.*\).doc'
\end{lstlisting}
\end{block}
\end{frame}

\section{command line}

\subsection{positional parameters}

\begin{frame}[fragile,containsverbatim]
\frametitle{Positional Parameters}
\begin{itemize}
	\item \lstverb|$*| treats all positional parameters as a whole string.
	\item \lstverb|$@| treats all positional parameters as an array.
	\item \lstverb|$0| is the script file itself.
	\item \lstverb|$1,$2,...| is the first, second, ... positional parameters.
	\item \lstverb|$#|: number of positional parameters.
\end{itemize}
\begin{block}{\centering Examples}
\begin{lstlisting}
#!/bin/bash
for i in {1..$#}; do
  echo ${!i}
done
\end{lstlisting}
\end{block}
\end{frame}

\begin{frame}[fragile,containsverbatim]
\frametitle{Exercise}
\begin{enumerate}
	\item Print all the positional arguments, one per line.
	\item Compute the product of all the positional arguments.
	\item If your arguments contain some float numbers, what to 
		do then?
\end{enumerate}
\end{frame}


\subsection{command-line options}

\begin{frame}[fragile,containsverbatim]
\frametitle{options}
\begin{itemize}
	\item \lstinline{./test.sh -a -b -c}\\short options 
		without argument (无参数短选项).
	\vspace{0.2in}
	\item \lstinline{./test.sh -abc}\\short options as above (同上).
	\vspace{0.2in}
	\item \lstinline{./test.sh -a arg -b -c}\\short options, where 
		\lstverb|-a arg| needs argument, while \lstverb|-b| and 
		\lstverb|-c| do not need argument (有/无参数短选项).
	\vspace{0.2in}
	\item \lstinline{./test.sh --a-long=arg --b-long}\\long options 
		where \lstverb|--a-long=arg| requires argument, while 
		\lstverb|--b-long| does not require argument (有/无参数长选项).
\end{itemize}
\end{frame}

\begin{frame}[fragile,containsverbatim]
\frametitle{\texttt{getopts}}
\begin{itemize}
	\item \lstverb|getopts| does not support long options.
\end{itemize}
\begin{block}{\centering Example}
\begin{lstlisting}
#!/bin/bash
# testoptions.sh
# ./testoptions.sh -a something -bc other-stuff
while getopts "a:bc" arg # colon indicates `a' requires argument
do
  case $arg in
  a)
    echo "a's arg: $OPTARG" ;;
  b)
    echo "b" ;;
  c)
    echo "c" ;;
  ?)
    echo "unknown argument"; exit 1 ;;
  esac
done
echo "The argument is at $OPTIND"
\end{lstlisting}
\end{block}
\end{frame}


\begin{frame}[fragile,containsverbatim]
\frametitle{\texttt{getopt}: supports long options}
\begin{lstlisting}
#!/bin/bash
# testlongopts.sh
# ./testlongopts.sh -a arg1 'arg2' --c-long 'wow!*\?' -cmore -b " very long "
TEMP=`getopt -o ab:c:: --long a-long,b-long:,c-long:: \
              -n 'testlongopts.sh' -- "$@"`
if [ $? != 0 ]; then echo "Terminating..." >&2 ; exit 1 ; fi
# `set' will reorder the parameter  
eval set -- "$TEMP"
while true ; do
  case "$1" in
    -a|--a-long) echo "Option a"; shift ;;
    -b|--b-long) echo "Option b, argument \`$2'" ; shift 2 ;;
    -c|--c-long)
      case "$2" in 
        "") echo "Option c, no argument"; shift 2 ;;
        *) echo "Option c, argument \`$2'" ; shift 2 ;;
      esac ;;
    --) shift ; break ;;
    *) echo "Internal error!" ; exit 1 ;;
  esac
done
echo "Remaining arguments: "
for arg do echo '--> '"\`$arg'" ; done
\end{lstlisting}
\end{frame}


\begin{frame}[fragile,containsverbatim]
\frametitle{Exercise}
\begin{enumerate}
	\item As a simple exercise, write a shell script that 
prints the file type for each file passed as an 
argument to the script. Here is an example:
\begin{lstlisting}[basicstyle=\footnotesize\ttfamily]
$ ./ftypes.sh ftypes.sh .emacs .bashrc public_html
ftypes.sh: Bourne-Again shell script text executable
.emacs: Lisp/Scheme program text
.bashrc: ASCII English text, with very long lines
public_html: symbolic link to `www'
\end{lstlisting}
\end{enumerate}
\end{frame}

\section{functions}

\begin{frame}[fragile,containsverbatim]
\frametitle{Functions: Definition}
\begin{block}{\centering Function defintion (函数定义)}
\begin{lstlisting}
# function func_name() { commands; }
function dir() {
  target=${1:-.}
  if [ -e $target ]; then
    echo "List Directories in $target: ";
    ls -l $target | awk '/^d/{print $NF}';
  else
    echo "$target not exists."
    return -1
  fi
}
\end{lstlisting}
\end{block}
\begin{block}{\centering Function call (函数调用)}
\begin{lstlisting}
#!/bin/bash
read -p "Input the directory name: " dirname
dir $dirname
\end{lstlisting}
\end{block}
\end{frame}


\begin{frame}[fragile,containsverbatim]
\frametitle{Variable Scopes (变量作用域)}
\begin{itemize}
	\item Local Variables (局部变量): \textbf{inside a function}
	\item Global Variables (全局变量): \lstinline{declare -g name=val}
	\item Environmental Variables (环境变量): \lstinline{export name=val}
\end{itemize}
\begin{block}{\centering Examples (示例)}
\begin{lstlisting}
#!/bin/bash
function1() {
  local func1var=20
  echo "Within function1, \$func1var = $func1var."
  function2
}
function2 () {
  echo "Within function2, \$func1var = $func1var."
}
function1
echo "Outside function, \$func1var = $func1var."
exit 0
\end{lstlisting}
\end{block}
\end{frame}


\begin{frame}[fragile,containsverbatim]
\frametitle{A little project: prime finder (1)}
\begin{lstlisting}
#!/bin/bash
# SCRIPT: primefactors.sh
# USAGE: primefactors.sh <Positive Integer>
# PURPOSE: Produces prime factors of a given number
####################################################
# Argument checking
####################################################
if [ $# -ne 1 ]; then
  echo "Usage: $0 <Positive Integer>"
  exit 1
fi
expr $1 + 1 &>/dev/null
if [ $? -ne 0 ]; then
  echo "Sorry, you supplied a non-numerical value."
  exit 1
fi
[ $1 -lt 2 ] && echo "Value < 2 are not prime numbers" && exit 1
num=$1
\end{lstlisting}
\end{frame}


\begin{frame}[fragile,containsverbatim]
\frametitle{A little project: prime finder (2)}
\begin{lstlisting}
########################################
# Functions
########################################
# script to find prime number
primenumber() {
  primenum=$1
  for ((cnt2=2;$((cnt2*cnt2))<=$primenum; cnt2++)); do
    [ $((primenum%cnt2)) -eq 0 ] && return 1
  done
  return 0
}
primefind() {
  primenumber $1 && echo "$1" && exit 0
  for ((cnt1=$2;cnt1<=$1;$cnt1++)); do
    primenumber $cnt1 && factorcheck $1 $cnt1 && break
  done
}
\end{lstlisting}
\end{frame}

\begin{frame}[fragile,containsverbatim]
\frametitle{A little project: prime finder (3)}
\begin{lstlisting}
factorcheck()
{
  prime=$2
  newnum=$1
  remainder=$((newnum%prime))

  if [ $remainder -eq 0 ]; then
    printf "%dx" $prime
    newnum=$((newnum/prime))
    primefind $newnum 2
    return
  else
    let prime++
    primefind $newnum $prime
  fi
}
\end{lstlisting}
\end{frame}

\begin{frame}[fragile,containsverbatim]
\frametitle{A little project: prime finder (4)}
\begin{lstlisting}
###############################################
# main
###############################################
echo -n "Prime Factor of $1: "
primefind $num 2
print "\b\n"
\end{lstlisting}
\end{frame}

\begin{frame}[fragile,containsverbatim]
\frametitle{Exercise}
Explain why the final output is blank.
\begin{lstlisting}[basicstyle=\scriptsize\ttfamily]
#!/bin/bash
OUTPUT="name1 ip ip status" # normally output of another command with multi line output

if [ -z "$OUTPUT" ]
then
        echo "Status WARN: No messages from Hacker"
        exit $STATE_WARNING
else
        echo "$OUTPUT"|while read NAME IP1 IP2 STATUS
        do
                if [ "$STATUS" != "Optimal" ]
                then
                        echo "CRIT: $NAME - $STATUS"
                        echo $((++XCODE))
                else
                        echo "OK: $NAME - $STATUS"
                fi
        done
fi

echo $XCODE
\end{lstlisting}
\end{frame}

\begin{frame}[fragile,containsverbatim]
\frametitle{Indexed Arrays}
\begin{itemize}
	\item \lstverb|declare -a arrayname|
	\item Only 1-dim array is allowed
	\item Array Assignment
	\begin{enumerate}
		\item \lstverb|array=(val1 val2 val3 ... valN)|
		\item \lstverb|array=([0]=val0 [1]=val1 ... [N]=valN)|
		\item \lstverb|array[0]=val0; array[1]=val1; ...|
	\end{enumerate}
	\item Size: \lstverb|${#array[@]}, ${#array[*]}|
	\item \lstverb|echo ${array[2]}|
	\item \lstverb|echo ${array[@]:3:2}| 
\end{itemize}
\end{frame}

\begin{frame}[fragile,containsverbatim]
\frametitle{Associative Arrays}
\begin{itemize}
	\item \lstverb|declare -A Arrayname|
	\item Array Definition
	\begin{enumerate}
		\item \lstverb|Array=([unix]=1 [windows]=2 [mac]=3)|
		\item \lstverb|Array[unix]=1; Array[windows]=2; Array[mac]=3|
	\end{enumerate}
	\item Iterate over all keys:
	\begin{lstlisting}
for key in "${!Array[@]}"; do echo $key; done
	\end{lstlisting}
\end{itemize}
\end{frame}


\begin{frame}[fragile,containsverbatim]
\frametitle{Summary}
\begin{itemize}
	\item Run script file: shebang (\lstverb|#!/bin/bash|)
	\item Variable assignment (\lstverb|declare|) and 
		destroy (\lstverb|unset|)
	\item Conditional statement (\lstverb|[]|, 
		\lstverb|[[]]|, \lstverb|test|)
	\item \lstverb|if...fi| statement
	\item \lstverb|for...do...done| statement
	\item \lstverb|case...in...esac| statement
	\item \lstverb|while...do...done| statement
	\item Function definition (\lstverb|func(){...}|)
	\item Positional parameters (\lstverb|$1,$*, $@|)
	\item \lstverb|getopts| statement 
\end{itemize}
\end{frame}

\end{CJK*}
\end{document}
