\documentclass[addpoints,12pt,answers]{exam}

\usepackage{color}
\usepackage{amsmath,amssymb}
\usepackage{listings}
\usepackage{CJKutf8}

\newcommand{\tf}[1][{}]{%
	\fillin[#1][0.25in]%
}
\newcommand{\tabincell}[2]{\begin{tabular}{@{}#1@{}}#2\end{tabular}}
\setlength{\parskip}{0.1 in}

\checkboxchar{$\Box$}
\checkedchar{$\blacksquare$}
\CorrectChoiceEmphasis{\color{red}}

\pagestyle{headandfoot}
\runningheadrule
\firstpageheader{BI296}{Quiz 1}{March 10, 2017}
\runningheader{BI296}
			{Quiz 1, Page \thepage\ of \numpages}
			{March 10, 2017}
\firstpagefooter{}{}{}
\runningfooter{}{}{}

\begin{document}
\begin{CJK*}{UTF8}{gbsn}

\begin{center}
\fbox{\fbox{\parbox{5.5in}{\centering
	如果预留空间不足,请将答案写到试卷背面。}}}
\end{center}

\vspace{0.1in}

\makebox[\textwidth]{学号: \enspace\hrulefill}

\vspace{0.1in}

\makebox[\textwidth]{姓名: \enspace\hrulefill}

\begin{center}
	\pointtable[h][pages]
\end{center}


\qformat{\textbf{Question \thequestion}\quad (\thepoints)\hfill}

\section{简答题}

\begin{questions}

\question[10]
将下列数值在不同的进制之间进行转换:
\begin{table}[h]
\center
\renewcommand\arraystretch{1.6}
\begin{tabular}{|c|c|c|c|c|}
\hline
\tabincell{c}{decimal\\(10进制)} & \tabincell{c}{binary\\(2进制)} 
& \tabincell{c}{octal\\(8进制)} & \tabincell{c}{hexadecimal\\(16进制)} 
& \tabincell{c}{septenary\\(7进制)} \\
\hline
235.50d & & & & \\
\hline
& 01101110 & & & \\
\hline
& & 0347 & & \\
\hline
& & & 0x34A2 & \\
\hline
\end{tabular}
\end{table}


\question[10]
IEEE表示单精度浮点数占据32位,分别用来表示符号位、阶码和尾数,其基本格式是:

\begin{table}[htbp]
\begin{tabular}{|l|l|l|}
\hline
$S$: 1-bit & $P$: 8-bit & $M$: 23-bit\\
\hline
\end{tabular}
\end{table}

其中$S$是符号位(0为正数,1为负数),$P$是阶码(移码,与补码除了符号位相反之外,
其它位完全相同,偏移量为127),$M$是尾数(原码)。其数值的计算公式是:
$$
(-1)^{S} \times 2^{P-127} \times 1.M
$$

这里所谓的尾数就是二进制浮点数的小数部分,其整数位为1。

那么,单精度数可以表示的数值范围是什么?其精度是什么?

\begin{solution}
\vspace{2in}

\end{solution}


\question[10]
请为乌有国国王出个主意:有128桶酒,其中有1桶是毒酒;48小时候要举行宴会,
人喝了毒酒之后的11-12小时内将会死去;国王决定用囚犯来试酒,而且不介意有
多少囚犯会死。你能否用最少的囚犯来测试哪一桶才是毒酒。请问至少需要多少
囚犯才能保证找出毒酒?最少可能会有多少囚犯因此而死?

\begin{solution}
\vspace{1in}
\end{solution}


\question
针对Linux中的文件权限管理,回答下列问题
\begin{parts}
\part[5]
我们用\lstinline{ls -l}查看文件\texttt{/usr/bin/locate},发现其权限为:
\begin{quote}
\texttt{rwx-{}-s-{}-x}
\end{quote}
请问这里包含哪些权限,分别阐明其功能。
\begin{solution}
\vspace{1in}
\end{solution}

\part[5]
对于目录\texttt{/var/tmp},其权限位是什么,有什么特殊含义?
\begin{solution}
\vspace{1in}
\end{solution}

\end{parts}

\question[10]
为什么32位操作系统最大只能支持4GB的内存?64位操作系统最大能支持的内存是多大?
\begin{solution}
\vspace{1in}
\end{solution}


\question[10]
能否让同组用户对某一个目录具有完全相同的权限?如果能,请写出实现的细节。
\begin{solution}
\vspace{1in}
\end{solution}

\end{questions}

\newpage

\section{选择题}

\begin{questions}
\question[2]
深度学习可能是目前最为火热的机器学习方法。在训练深度神经网络时,需要用到大量的矩阵乘积计算。
假设我们有三个稠密矩阵$\mathbf{A} \in \mathbb{R}^{m \times n}, \mathbf{B} \in \mathbb{R}^{n 
\times p}$和$\mathbf{C} \in \mathbb{R}^{p \times q}$,且$m < n < p < q$下面哪个乘积
的组合计算能最快得到结果?在边上注明原因。
\begin{choices}
\choice $\mathbf{A(BC)}$
\choice $\mathbf{(AB)C}$
\choice $\mathbf{(AC)B}$
\choice 没有差别
\end{choices}
\answerline

\question[2]
下面哪个系统不是类UNIX操作系统?
\begin{choices}
\choice MacOS
\choice Ubuntu
\choice Fedora
\choice Windows
\end{choices}
\answerline

\question[2]
下面的名人当中,谁才是GNU项目的创始人?
\begin{choices}
\choice Linus Torvalds
\choice Ken Thompson
\choice Dennis Ritchie
\choice Richard Stallman
\end{choices}
\answerline

\question[2]
下面列表中,哪个不是Linux内核的特征?
\begin{choices}
	\choice free and open-source
	\choice	multi-task and multi-user
	\choice under GNU GPL
	\choice All the distributions are free of charge.
\end{choices}
\answerline

\question[2]
下面哪些命令,可以输出文件的3-20行内容?
\begin{choices}
	\choice \lstinline{sed '3,+20p' filename}
	\choice \lstinline{head -20 filename | tail -n 18}
	\choice \lstinline!awk '3,20{ print $0 }' filename!
	\choice \lstinline{cat filename | grep 3 20}
\end{choices}
\answerline

\question[2]
下列数值最大的是?
\begin{choices}
	\choice (10010101)\_2
	\choice (227)\_8
	\choice (96)\_16
	\choice (143)\_10
\end{choices}
\answerline

\question[2]
如果78+78=123成立,则应该采用的是哪种进制?
\begin{choices}
	\choice 11
	\choice 12
	\choice 13
	\choice 14
	\choice 15
\end{choices}
\answerline


\question[2]
八进制数256,转化位7进制数是?
\begin{choices}
	\choice 356
	\choice 346
	\choice 336
	\choice 338
\end{choices}
\answerline

\question[2]
n从1开始,每个操作可以选择对n加1或者对n加倍,若想获得整数2013,最少需要多少次操作?
\begin{choices}
	\choice 24
	\choice 21
	\choice 18
	\choice 19
\end{choices}
\answerline


\question[2]
CPU的高速缓存指的是?
\begin{choices}
	\choice CPU多个内核之间传输数据的通道
	\choice 加快CPU访问内存速度的存储器
	\choice 加快磁盘读写速度的存储器
	\choice 加快计算机周边设备与内存传输数据的部件
\end{choices}
\answerline
\end{questions}

\newpage

\section{填空题}

\begin{questions}
	\question[2] 16-bit无符号整数的取值范围是\fillin[]; 有符号整数的取值范围是\fillin[][1in]。
	\question[2] 某台主机的IP为192.168.1.112, 子网掩码是255.255.255.192,则其所在的子网的IP范围是\fillin[][1in]。
	\question[2] Socket连接通信位于OSI网络协议结构的\fillin[]层。
	\question[2] 文件系统目录中\texttt{/usr}中的\texttt{usr}是\fillin[][1in]的缩写。
	\question[2] 文件\texttt{/dev/sda}是特殊的\fillin[][1in]文件;而\texttt{/dev/null}是特殊的\fillin[][1in]文件。
	\question[2] 如果需要将某用户新文件的默认权限设置为600,则应该运行\fillin[][1in]。
	\question[2] 需要寻找所有根目录下的所有Named pipe文件,需要运行命令\fillin[][1in]。
	\question[2] 要输出库函数\texttt{vprintf}的帮助信息,需要运行\fillin[][1in];该帮助为手册的第\fillin[]节。
	\question[2] 命令\texttt{ls -l}默认输出的时间戳是\fillin[],其意义是\fillin[][2in]。
	\question[2] 目录\texttt{/etc}存放的是\fillin[][2in];其中\texttt{/etc/shadow}中存放的是\fillin[][2in]。
\end{questions}

\end{CJK*}
\end{document}
